% For my printer and adobe version, I have to set up the margins as follows ans select "print orignal" or whatever it says.
\documentclass[12pt,letterpaper,oneside]{report}

\usepackage[top=0.5625in, bottom=1.0in, left=1.0625in, right=0.9375in]{geometry}
% Acutal dimensions below, those above are hacked to fix 1/16in left & down shift in my printer.
%\usepackage[top=0.5625in, bottom=1.0in, left=1.0625in, right=0.9375in]{geometry}
\usepackage{graphicx}
\usepackage{epstopdf}
\usepackage{tikz}
\usepackage{helvet}
\renewcommand{\familydefault}{\sfdefault}

\begin{document}
\thispagestyle{empty}

{\footnotesize \centerline{\textbf{DEPARTMENT OF THE ARMY}}}
{\scriptsize \centerline{\textbf{ORGANIZATIONAL NAME/TITLE}}}
{\scriptsize \centerline{\textbf{STANDARDIZED STREET ADDRESS}}}
{\scriptsize \centerline{\textbf{CITY STATE 12345-1234}}}

{\begin{tikzpicture}[remember picture, overlay]
% don't ask why the picture has to be shifted to these somwewhat random coordinates - I don't know. But this is what gets the seal to 0.5in from top and left.
     \node [xshift=-0.25in,yshift=0.3125in] at (0,0)
     {\includegraphics[height=1in]{DODb1.eps}};
  \end{tikzpicture}}

\setlength{\parindent}{14.82mm}
\setlength{\parskip}{-8pt}
{\tiny REPLY TO}

\setlength{\parindent}{14.82mm}
\setlength{\parskip}{-8pt}
{\tiny ATTENTION OF}

\setlength{\parskip}{12pt}
\noindent
OFFICE SYMBOL \hfill DATE

\begin{raggedright}
\setlength{\parskip}{24pt}
\noindent
MEMORANDUM FOR U.S. Army Command and General Staff College (ATZL), 100 stimson Avenue, Ft Leavenworth, KS 66027-1352

\setlength{\parskip}{12pt}
\noindent
SUBJECT: Using and Preparing a Memorandum

\setlength{\parskip}{24pt}
\noindent
1.\hphantom{.} See paragraph 2-2 (of this regulation) for when to use a memorandum.

\setlength{\parskip}{12pt}
\noindent
2.\hphantom{.} Single space the text and double space between paragraphs and subparagraphs. Insert two blank spaces after ending punctuation (period and question mark). Insert two blank spaces after a colon. When numbering subparagraphs, insert two blank spaces after parentheses.

\setlength{\parskip}{12pt}
\noindent
3.\hphantom{.} When a memorandum has more than one paragraph, number the paragraphs consecutively. When paragraphs are subdivided, designate first subdivisions using lowercase letters of the alphabet and indent 1/4 inch as shown below.

\setlength{\parskip}{12pt}
\setlength\parindent{0.25in}
a.\hphantom{x} When a paragraph is subdivided, it must has at least two subparagraphs.

\setlength{\parskip}{12pt}
\setlength\parindent{0.25in}
b.\hphantom{x} If there is a subparagraph "a," there must be a subparagraph "b."

\setlength{\parskip}{12pt}
\setlength\parindent{0.5in}
(1)\hphantom{x} Designate second subdivisions by numbers in parenteses; for example, (1), (2), and (3) and indent 1/2 inch as shown.

\setlength{\parskip}{12pt}
\setlength\parindent{0.5in}
(2)\hphantom{x} Do not subdivide beyond the third subdivision.

\setlength{\parskip}{12pt}
\setlength\parindent{0.5in}
(a)\hphantom{x} Do not indent any futher than the second subdivision.

\setlength{\parskip}{12pt}
\setlength\parindent{0.5in}
(b)\hphantom{x} Use (a), (b), (c), and so forth at this level.

\setlength{\parskip}{48pt}
\setlength\parindent{3.25in}
JOHN W. SMITH

\setlength{\parskip}{0pt}
\setlength\parindent{3.25in}
Colonel, GS

\setlength{\parskip}{0pt}
\setlength\parindent{3.25in}
Chief of Staff

\end{raggedright}
\end{document}
